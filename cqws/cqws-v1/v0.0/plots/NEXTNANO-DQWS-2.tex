% !TeX document-id = {bc853b60-2dbb-4300-accd-3581ab2279d9}
% !TeX TXS-program:compile = txs:///pdflatex/[--shell-escape]
\RequirePackage{fix-cm}
\documentclass[border=0pt]{standalone}
\usepackage[utf8]{inputenc}
\usepackage{lmodern}
\usepackage[T1]{fontenc}
\usepackage{graphics}
\usepackage{xcolor}
\usepackage{pgffor} 
\usepackage{scalerel}
\usepackage{tikz,filecontents, pgfplots}
\pgfplotsset{compat=1.6}
\usepackage{pgfplotstable}
\usetikzlibrary{arrows,
	pgfplots.groupplots,
	arrows.meta,
	decorations.pathmorphing,
	calc,%
	decorations.pathmorphing,%
	decorations.markings,
	fadings,%
	shadings,%
	positioning,
	spy,
	shapes,
	shapes.geometric,
	shapes.arrows,
	fit,
	plotmarks,
	intersections}

\tikzset{
	arrow/.style={-{Stealth[scale=1]}, line width=0.75pt},
}

% Colors
\definecolor{c1}{rgb}{1,0,0}
\definecolor{c2}{rgb}{.05,0.14,0.81}
\definecolor{c3}{rgb}{.074,0.572,0.34}
\definecolor{c4}{rgb}{.59,0.80,0.84}
% Constants
\def\xmin{0}
\def\xmax{56}
\def\ymine{1.50}
\def\ymaxe{1.69}
\def\yminh{-0.098}
\def\ymaxh{0.20}

%
\def\eu{0.04912}
\def\ed{0.05765}

\begin{document}
\begin{tikzpicture}[spy using outlines={circle,black, ultra thick,connect spies}]
\pgfplotsset{
	every axis/.append style = {
		line width = 0.5pt,
		tick style = {line width=0.5pt,black},
		major tick length = 0.15cm,
		minor tick length = 0.075cm,
		%axis line style = thick,
	}
}


%============================================= Plot ==========================================	
\begin{axis}[%
ymin=-0.02, ymax=0.25,
xmin=-2.5,  xmax=58,
height=6cm,
width=7cm,
minor x tick num=1,
minor y tick num=1,
ytick pos=left,
xtick pos=left,
ytick={-0.05,-0.00,0.05,0.10,0.15,0.20,0.25,0.30},
xtick={-5,0,  5., 10., 15., 20., 25., 30., 35., 40., 45., 50., 55., 60.},
yticklabel style = {font=\fontsize{6}{1}\selectfont},
xticklabel style = {font=\fontsize{6}{1}\selectfont},
tick label style = {font=\fontsize{6}{1}\selectfont},
 y tick label style={
	/pgf/number format/.cd,
	fixed,
	fixed zerofill,
	precision=2
},
%======================================  legend ============================================
legend style = {at={(0.015,0.98)},
	anchor=north west,
	nodes={inner sep=1pt},
	cells={anchor=center},
	draw=none, 
	fill=none,
	font={\fontsize{4}{1}\selectfont}, 
	line width =0.75pt},
legend cell align=left,
%=========================================Labels ======================== =====================
xlabel ={\fontsize{7}{1}\selectfont Z  [m] },
ylabel ={\fontsize{7}{1}\selectfont  Energ\'ia [eV]},
every axis y label/.style= {at={(axis description cs:-0.15,0.5)},rotate=90},
every y tick scale label/.style={at={(axis description cs:-0.15,0.42)},
	rotate=90,
	inner sep=0pt},
every x tick scale label/.style={at={(axis description cs:0.65,-0.15)},
	rotate=0,
	inner sep=0pt},
tick scale binop = \times,
]

%===========================================Add Data ===========================================
\pgfplotstableread[col sep=comma]{../DATA/Potential_Profile.dat}{\vp};
\pgfplotstableread[col sep=comma]{../DATA/PSIelectron_data.dat}{\psie};


%\addplot[mark=none, black, dashed, line width = 1pt] coordinates {(815.932,-0.002) (815.932,0.004)};

%0.016503
%=================================== Plot ======================================


\addplot[black,line width = 0.35pt,forget plot] table[x index=0,y index=1]{\vp};
\addplot[c1,densely dashed,line width =0.5pt] coordinates {(\xmin,\eu) (\xmax,\eu)};
\addlegendentry[minimum height=0.25cm,yshift = -0.3mm]{e$_{\scaleto{1}{2pt}}$}
\addplot[c2,densely dashed,line width =0.5pt] coordinates {(\xmin,\ed) (\xmax,\ed)};      
\addlegendentry[yshift = -0.3mm]{e$_{\scaleto{2}{2pt}}$};
\addplot[smooth,c1,line width =0.5pt] table[x index=0,y index=1]{\psie};  
\addlegendentry[yshift = -0mm]{$\psi^{2}_{\scaleto{e}{2pt}_{\scaleto{1}{2pt}}}$};
%\addplot[smooth,c2,line width =0.5pt] table[x index=0,y index=2]{\psie};
%\addlegendentry{ $\psi^{2}_{\scaleto{e}{2pt}_{\scaleto{2}{2pt}}}$};

\end{axis}

\end{tikzpicture}

\end{document}